\documentclass{uetgraduation}
\begin{document}
\studentname{Hoàng Trung Dũng}
\title{Nghiên cứu phương pháp chống nhiễu cho mạng truyền thông tán xạ ngược sử dụng phương pháp học sâu tăng cường}
\documenttype{Khóa luận tốt nghiệp đại học hệ chính quy}
\major{Công nghệ thông tin}
\year{2024}
\supervisor{TS. Nguyễn Ngọc Tân}
\makecovers
\begin{preamble}{Tóm tắt}
\textbf{Tóm tắt:} Trong bối cảnh UAVs được sử dụng như các thiết bị gây nhiễu trong mạng truyền thông không dây, 
việc duy trì hiệu quả truyền thông và tối ưu hóa mức tiêu thụ năng lượng của transmitter trở nên cần thiết hơn 
bao giờ hết. Khoá luận này tập trung vào việc phát triển một mô hình học sâu tăng cường để điều khiển động thái 
hoạt động của transmitter, giúp cải thiện khả năng thích ứng với các điều kiện nhiễu biến đổi và tối ưu hóa việc 
sử dụng năng lượng.

Phương pháp được đề xuất sử dụng thuật toán học tăng cường để tự động hóa việc ra quyết định trong việc chọn lựa 
giữa ba hoạt động chính: thu hoạch năng lượng từ tín hiệu gây nhiễu, truyền tin trong trạng thái hoạt động và điều 
chỉnh tốc độ truyền (rate adaptation). Mô hình DRL được huấn luyện để tối đa hóa một hàm lợi ích, tính toán dựa trên 
hiệu quả truyền thông và hiệu quả năng lượng, qua đó giúp transmitter phản ứng linh hoạt và hiệu quả hơn với môi trường 
nhiễu do UAV tạo ra.

Kết quả thử nghiệm cho thấy mô hình DRL có khả năng đáng kể trong việc cải thiện hiệu suất truyền thông trong điều kiện 
bị gây nhiễu bởi UAV, đồng thời nâng cao hiệu quả sử dụng năng lượng so với các phương pháp điều khiển truyền thống. Các 
kết quả này không những chỉ ra tiềm năng của học sâu tăng cường trong việc quản lý hoạt động của transmitter mà còn mở 
rộng khả năng ứng dụng của nó trong các mạng không dây chịu nhiều nhiễu và yêu cầu cao về bảo mật và độ tin cậy.

\textit{\textbf{Từ khóa:} UAV, Nhiễu, Học sâu tăng cường, Tán xạ ngược, Thu hoạch năng lượng.}
\end{preamble}
\end{document}