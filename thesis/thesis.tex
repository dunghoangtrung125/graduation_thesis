\documentclass{uetgraduation}
\usepackage{graphicx}
\graphicspath{{graph/}}
\begin{document}
\studentname{Hoàng Trung Dũng}
\title{Nghiên cứu phương pháp chống nhiễu cho mạng truyền thông tán xạ ngược sử dụng phương pháp học sâu tăng cường}
\documenttype{Đồ án tốt nghiệp đại học hệ chính quy}
\major{Công nghệ thông tin}
\year{2024}
\supervisor{TS. Nguyễn Ngọc Tân}
\makecovers
% Tóm tắt
\begin{preamble}{Tóm tắt}
\textbf{Tóm tắt:} Truyền thông không dây đã và đang đóng vai trò vô cùng quan trọng trong cuộc sống con người. Tuy
nhiên phương pháp truyền thông này lại rất dễ bị tấn công gây nhiễu do tín hiệu vô tuyến phát sóng trong không gian mở.
Thêm vào đó, với sự phát triển của UAV (thiết bị bay không người lái) với khả năng cung cấp đường truyền tầm nhìn thẳng
(LoS) và hệ số suy giảm đường truyền thấp đã hỗ trợ cho việc tấn công đối với kết nối không dây. Trong khoá luận tốt nghiệp
này, em muốn trình bày một phương án chống nhiễu cho mạng truyền thông không dây, sử dụng học tăng cường sâu, kết hợp với
kỹ thuật tán xạ ngược và thu hoạch năng lượng để không những chống lại mà còn tận dụng được tín hiệu gây nhiễu từ UAV 
để nâng cao hiệu suất của hệ thống truyền thông không dây.

\textit{\textbf{Từ khóa:} Truyền thông không dây, Nhiễu, UAV, Học tăng cường sâu, Tán xạ ngược, Thu năng lượng.}
\end{preamble}
% Lời cảm ơn
\begin{preamble}{Lời cảm ơn}
    Đầu tiên, cho phép em gửi lời cảm ơn đến các thầy, cô giáo trường Đại học Công nghệ - Đại học
    Quốc Gia Hà Nội đã luôn tận tình chỉ bảo và tạo điều kiện trong suốt quá trình em học
    tập tại trường.
    
    Em xin gửi lời cảm ơn sâu sắc đến thầy giáo TS. Nguyễn Ngọc Tân đã tận tình
    hướng dẫn và đóng góp ý kiến quý báu trong suốt quá trình thực hiện khóa luận tốt
    nghiệp của em.
    
    Cuối cùng em xin gửi lời cảm ơn đến gia đình của mình, nơi đã luôn là nguồn động lực cho em
    trong suốt thời gian vừa qua.
    
    Em xin chân thành cảm ơn.
\end{preamble}
% Lời cam đoan
\begin{preamble}{Lời cam đoan}
Tôi xin cam đoan rằng mọi kết quả trình bày trong khóa luận đều do tôi thực hiện
dưới sự hướng dẫn của TS. Nguyễn Ngọc Tân.

Tất cả các tham khảo nghiên cứu liên quan đều nêu rõ nguồn gốc một cách rõ ràng từ 
danh mục tài liệu tham khảo trong khóa luận. Khóa luận không sao chép tài liệu, 
công trình nghiên cứu từ người khác mà không có rõ về mặt tài liệu tham khảo.

Các thông kê, các kết quả trình bày khóa luận đều là tự thực nghiệm khi chạy chương trình. Nếu tôi sai 
tôi hoàn toàn chịu trách nhiệm theo quy định của trường Đại học Công Nghệ - Đại học Quốc Gia Hà Nội.

\begin{flushright}
    Hà Nội, tháng 12 năm 2024

    \vspace{45pt}
    Hoàng Trung Dũng
\end{flushright}
\end{preamble}

% Muc luc
\begin{contentlisting}

\tableofcontents
\listoffigures
\listoftables

\begin{contentlistingsection}{Các từ viết tắt}
    UAV: unmanned aerial vehicle -- Thiết bị bay không người lái

    MDP: Markov decision process

    DRL: deep reinforcement learning -- Học tăng cường sâu.
    
    DQN: deep q network -- Mạng sâu Q.

    HTT: harvest then transmit -- Chiến lược thu năng lượng để truyền tin
\end{contentlistingsection}

\end{contentlisting}

% Chapter 1
\chapter{Đặt vấn đề}
Phần còn lại của khoá luận tốt nghiệp được trình bày theo thứ tự như sau

% Chapter 2
\chapter{Cơ sở lý thuyết}

\section{Ambient backscatter}

\section{Bài toán gây nhiễu}
\section{Bài toán gây nhiễu bằng UAV}
\section{Kĩ thuật giải quyết bài toán gây nhiễu}

% Chapter 3
\chapter{Đề xuất phương pháp giải quyết bài toán gây nhiễu từ UAV}

% Chapter 4
\chapter{Thiết lập mô phỏng}

% Tài liệu tham khảo
\begin{thebibliography}{9}
\begin{bibsection}{Tiếng Anh}
\bibitem{drlbook}
    Dinh Thai Hoang, Nguyen Van Huynh, Diep N. Nguyen, Ekram Hossain, Dusit Niyato,
    \textit{Deep Reinforcement Learning for Wireless Communications and Networking: Theory, 
    Applications and Implementation},
    \textit{Wiley-IEEE Press},
    2023, pp. 37-163.
\end{bibsection}
\end{thebibliography}
\end{document}